\problem{Powersets have cardinality $2^n$}

Let $S$ be any finite set, and suppose $x \not\in S$. Let $K = S \cup {x}$.

\begin{enumerate}
    \item Prove that $\mathcal{P}(K)$ is the disjoint union of $\mathcal{P}(S)$ and $X = {T \subseteq K : x \in T}$
    \item Prove that every element of X is the union of a subset of $S$ with ${x}$, and if you take different subsets of S you get different elements of $X$. Argue that, therefore, $X$ has the same number of elements as $\mathcal{P}(S)$ 
\end{enumerate}

\solution

\part 
The second theorem below is proven while proving the first, so I've combined both here.

\begin{theorem}
    $\mathcal{P}(K)$ is the disjoint union of $\mathcal{P}(S)$ and $X = \{T \subseteq K : x \in T\}$ 
    (i.e. $\mathcal{P}(K) = \mathcal{P}(S) \cup X$ and $\mathcal{P}(S) \cap X = \varnothing$)
\end{theorem}
\begin{theorem}
    Every element of X is the union of a subset of $S$ with $\{x\}$, and that if you take different subsets of S you get different elements of $X$. Argue that, therefore, $X$ has the same number of elements as $\mathcal{P}(S)$ 
\end{theorem}

\begin{proof}
    The powerset, $\mathcal{P}(U)$ is defined as the set of all subsets of of the set $U$.
    Each subset (of $U$, say) is defined as the set $V$ where each element $v \in V$ is also in $U$.
    So, the powerset definition, in total, is $\mathcal{P}(U) = \{B | b \in U \text{ for all } b \in B\}$.
    As a new element, $x$ is added to $S$ to create $K$, the new powerset must now include subsets containing $x$ (since subsets are any set which contains elements in the parent set), which are based on the original elements of $\mathcal{P}(S)$, such that $X = \{B \cup \{x\} | B \in \mathcal{P}(S)\}$.
    The powerset is then the union of these sets ($\mathcal{P}(S) \text{and} X$), by definition.
    It is very simple to see that these sets ($\mathcal{P}(S) \text{and} X$) have the same cardinality, from the definition of $X$. (This shows that the cardinality of $\mathcal{P}(K)$ is twice that of $\mathcal{P}(S)$).
    It is also very simple to see that the intersection of $\mathcal{P}(S) \text{and} X$ is the null set, because every set in $X$ contains $x$, and every set in $\mathcal{P}(S)$ does not.
\end{proof}

\part 
Provided above as part of the same proof.
